\chapter{Testowanie i podsumowanie}

\section{Testowanie systemu}

Testowanie aplikacji odbyło się na kilku poziomach: jednostkowym, integracyjnym oraz użytkowym.

\subsection{Testy jednostkowe}
Testy jednostkowe skupiały się na poprawnym działaniu poszczególnych metod klas modelu oraz DAO. Szczególną uwagę zwrócono na poprawność:

\begin{itemize}
\item walidacji danych wejściowych,
\item działania metod CRUD,
\item obsługi wyjątków bazodanowych i walidacyjnych.
\end{itemize}

Wszystkie testy jednostkowe zostały przeprowadzone ręcznie poprzez sprawdzanie poszczególnych funkcjonalności w środowisku IDE (IntelliJ IDEA).

\subsection{Testy integracyjne}
Testy integracyjne dotyczyły komunikacji między warstwą aplikacji (DAO) a bazą danych PostgreSQL. Testowano poprawność zapytań SQL oraz operacji CRUD realizowanych z poziomu aplikacji. Wszystkie operacje (dodawanie, edycja, usuwanie, filtrowanie) przebiegły pomyślnie, a wyniki testów potwierdziły poprawne funkcjonowanie aplikacji.

\subsection{Testy użytkowe}
Testy użytkowe zostały przeprowadzone z punktu widzenia końcowego użytkownika aplikacji. Testy obejmowały:

\begin{itemize}
\item logowanie i autoryzację użytkowników,
\item rezerwację wizyt z walidacją konfliktów terminów,
\item zarządzanie danymi pacjentów, lekarzy i wizyt przez sekretariat,
\item obsługę sytuacji wyjątkowych (np. błędny login lub zajęty termin wizyty).
\end{itemize}

Aplikacja wypadła pozytywnie, co potwierdziło intuicyjność oraz poprawność działania wszystkich zaimplementowanych funkcjonalności.

\section{Podsumowanie}

Głównym celem projektu było stworzenie funkcjonalnej aplikacji umożliwiającej zarządzanie wizytami lekarskimi. Projekt został zrealizowany zgodnie z wymaganiami, z wykorzystaniem technologii Java oraz Swing, z relacyjną bazą danych PostgreSQL. Zastosowano podejście obiektowe, które zapewniło czytelność oraz łatwość utrzymania kodu.

W przyszłości projekt może zostać rozbudowany o dodatkowe funkcjonalności, takie jak integracja z kalendarzem internetowym czy możliwość eksportu danych do popularnych formatów (np. CSV lub XLS).

Kod źródłowy projektu jest dostępny publicznie w repozytorium:
\begin{center}
\href{https://github.com/YevhenMarchak/Aplikacja_do_zarzadzania_wizytami_lekarskimi.git}{\texttt{github.com/YevhenMarchak/Aplikacja\_do\_zarzadzania\_wizytami\_lekarskimi}}
\end{center}