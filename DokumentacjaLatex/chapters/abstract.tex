\chapter*{Streszczenie}
\addcontentsline{toc}{chapter}{Streszczenie}

Celem niniejszego projektu było zaprojektowanie i zaimplementowanie aplikacji desktopowej do kompleksowego zarządzania wizytami lekarskimi. System umożliwia rejestrację i logowanie pacjentów oraz sekretariatu, przeglądanie i edycję danych pacjentów, lekarzy oraz historii wizyt. 

Zastosowano język Java 17, bibliotekę Swing do stworzenia graficznego interfejsu użytkownika oraz bazę danych PostgreSQL z obsługą poprzez JDBC. Projekt uwzględnia warstwową architekturę aplikacji, walidację danych oraz obsługę wyjątków.

Efektem pracy jest funkcjonalna i intuicyjna aplikacja wspomagająca zarządzanie procesem umawiania wizyt w placówce medycznej.

\vspace{1cm}

\chapter*{Abstract}
\addcontentsline{toc}{chapter}{Abstract}

The aim of this project was to design and implement a desktop application for comprehensive management of medical appointments. The system allows for registration and login of patients and secretaries, as well as browsing and editing data of patients, doctors, and visit history.

The project uses Java 17, the Swing library for GUI creation, and PostgreSQL with JDBC for database handling. It follows a layered architecture and includes data validation and exception handling.

The result is a functional and user-friendly application that supports the scheduling of appointments in a medical facility.
