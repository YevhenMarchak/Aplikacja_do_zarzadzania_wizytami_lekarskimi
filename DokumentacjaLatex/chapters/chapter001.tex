\chapter{Opis założeń projektu}

\section{Cel projektu}
Celem niniejszego projektu było stworzenie aplikacji desktopowej umożliwiającej kompleksowe zarządzanie wizytami lekarskimi. Aplikacja pozwala na przegląd, dodawanie, edycję oraz usuwanie informacji o pacjentach, lekarzach i wizytach.

\section{Zakres projektu}
Projekt obejmuje implementację:
\begin{itemize}
\item interfejsu graficznego w technologii \textbf{Swing},
\item komunikacji z relacyjną bazą danych PostgreSQL przez \textbf{JDBC},
\item systemu autoryzacji użytkowników (sekretariat, pacjent),
\item podstawowych operacji \textbf{CRUD} na danych,
\item walidacji, filtrowania, sortowania i obsługi wyjątków.
\end{itemize}

\section{Uzasadnienie wyboru tematu}
Temat został wybrany ze względu na jego praktyczne zastosowanie oraz popularność w instytucjach medycznych. Projekt umożliwił studentowi rozwój umiejętności z zakresu projektowania aplikacji wielowarstwowych, obsługi baz danych oraz projektowania GUI.

\section{Technologie}
W projekcie zastosowano:
\begin{itemize}
\item język programowania \textbf{Java 17},
\item bibliotekę \textbf{Swing} do interfejsu graficznego,
\item \textbf{PostgreSQL} jako system zarządzania bazą danych,
\item narzędzie \textbf{LaTeX} do przygotowania dokumentacji.
\end{itemize}

\section{Struktura pracy}
Praca składa się z sześciu głównych rozdziałów:
\begin{itemize}
\item Opis założeń projektu,
\item Opis struktury projektu,
\item Harmonogram realizacji projektu,
\item Prezentacja warstwy użytkowej projektu,
\item Implementacja projektu,
\item Testowanie i podsumowanie,
\end{itemize}

